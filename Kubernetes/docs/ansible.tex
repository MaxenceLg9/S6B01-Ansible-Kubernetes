\documentclass{report}
\usepackage[T1]{fontenc}
\usepackage{graphicx}
\usepackage[table]{xcolor}
\usepackage[french]{babel}
\usepackage{titlesec}
\usepackage[a4paper]{geometry}
\usepackage{listings}
\usepackage[utf8]{inputenc}
\usepackage{lmodern}
\usepackage{babel}
\usepackage{stix}
\usepackage{minted}
\usepackage{fontspec}
\usepackage{tcolorbox}
\usepackage{hyperref}
\usepackage{titling}
\usepackage{enumitem}
\usepackage{fancyvrb}
\usepackage{tikz}
\usepackage{changepage}
\usepackage{tabularx}
\usepackage{float}
\usepackage{amsmath, amssymb}

\setmainfont{Calibri}



\setlist[itemize]{label=\large\textbullet}


\definecolor{azure}{rgb}{0.2, 0.7, 1.0}
\definecolor{bggray}{gray}{0.95}

\setlength{\parindent}{0pt}

\hypersetup{
	colorlinks=true,
	linkcolor=purple,
	filecolor=magenta,      
	urlcolor=blue,
	pdfborder={0 0 1}
}

\titleformat{\chapter}[block]
{\normalfont\LARGE\bfseries} % Style: large bold text
{\thechapter}                % Keep chapter number (remove if unwanted)
{1em}                        % Spacing between number and title
{}     

\urlstyle{same}

\geometry{width=18cm}
\geometry{a4paper}

\lstset{
	basicstyle=\ttfamily\small, % typewriter font
	keywordstyle=\color{blue}\bfseries, % keywords
	commentstyle=\color{green!50!black}\itshape, % comments
	stringstyle=\color{red}, % strings
	showstringspaces=false,
	numbers=none, % line numbers on the left
	numberstyle=\tiny\color{gray},
	backgroundcolor=\color{bggray},
	breaklines=true,
	frame=none,
	tabsize=4
}

\lstdefinelanguage{Rust}{
    keywords={fn, let, mut, if, else, match, impl, struct, enum, use, pub},
    sensitive=true,
    comment=[l]{//},
    morecomment=[s]{/*}{*/},
    morestring=[b]",
    basicstyle=\ttfamily\small,
    keywordstyle=\color{blue},
    commentstyle=\color{gray},
    stringstyle=\color{red},
}

\newenvironment{terminal}[1]{%
	\Verbatim[frame=none, numbers=none,label={#1}, breaklines, breakanywhere,tabsize=4,breaksymbol=, breakanywheresymbolpre=,backgroundcolor=bggray]%
}{%
	\endVerbatim
}

\renewcommand{\thechapter}{\Roman{chapter}}
\renewcommand{\thesection}{\thechapter.\Alph{section}}
\renewcommand{\thesubsection}{\thesection.\arabic{subsection}}
\renewcommand{\thesubsubsection}{\thesubsection.\alph{subsubsection}}


\pretitle{%
	\begin{center}
		\LARGE
		\includegraphics[width=6cm,height=2cm]{../../../../../../../Format/logo-UT-site.png}\\[\bigskipamount]
	}
	\posttitle{\end{center}}

\setcounter{secnumdepth}{4}
\setcounter{tocdepth}{3} 

\title{\Huge{\bfseries S6.B.01\\SAE\\Ansible}}
\date{\today}
\author{Maxence Lagourgue}

\begin{document}
	
	\maketitle
	\tableofcontents
	
	\chapter{Introduction}
	
	As part of the penultimate phase of the project realised for the 3thd Year, we have to deploy our previous infrastructure made in phase 4 with Ansible, to improve fast deployment and then high availability.
	
	The main components we have to deploy are:
	\begin{enumerate}
	\item k3s cluster
	\item Rancher through helm
	\item Create RKE2 cluster with workers and master node
	\item Create deployment / secrets / volumes / etc... with kubectl
	\end{enumerate}
	
	\chapter{Requirements}
	
	To be able to deploy Kubernetes artefacts, we firstly have to install some modules and roles dedicated for these tasks.
	
	\begin{itemize}
	\item ansible-galaxy collection install kubernetes.core
	\item ansible-galaxy role install lablabs.rke2
	\end{itemize}
	
	We also have to configure the users on the machines to be able to connect through SSH without typing a password, and to be able to use sudo without password too.
	
	On each machine we have to add an user, install ssh and modify the sudoers file.
	\begin{lstlisting}[language=Bash,caption={}]
	adduser sisyphe
	apt install ssh
	echo "sisyphe ALL=(ALL:ALL) NOPASSWD: ALL >> /etc/sudoers
	\end{lstlisting}
	
	We can now copy the public key to the host machine.
	\begin{lstlisting}[language=Bash,caption={}]
	ssh-copy-id -i  SSH/sisyphe_key.pub sisyphe@10.0.1.13
	\end{lstlisting}
	
	\chapter{Issues}
	
	During this phase, I encountered various issues. However, I have been able to run the infrastructure with ansible with special conditions which are:
	
	\begin{itemize}
	\item Making the infrastructure by hand, and waiting for the node to register, even if there was error.
	\item Creating the infrastructure by hand, but creating the cluster through ansible, fine too, we have to wait for the node to register itself.
	\item I now have to test if making all the infrastructure by Ansible works, still waiting for the node to register as it takes time.
	\end{itemize}
	
	I think, and I guess that if I restart rke2-server.service before it finishes, all the infrastructure crashes, so my best option is just to wait.
	
	For tomorrow, I have to test the full Infrastructure setup with Ansible.
	
	
\end{document}
